\section*{Summary}
Revival of Balochistan Water Resources Program (BWRP)" is an EU-funded program to contribute to the transition of rural irrigated agriculture and livestock farming systems in the arid regions of Balochistan towards lower water use and sustainable agricultural and livestock farming systems.
\\

The BWRP's TA component is being implemented jointly by Landell Mills and International Water Management Institute (IWMI). One of the objectives of the project is to strengthen governance of water resources and rangelands in Balochistan. This will include reinforcing the organizational and administrative capacity of water authorities to adopt strategic decisions, along with improving the policy and legal framework for water and land use in the agricultural and livestock sector.
\\

Integrated water resource management (IWRM) will also be developed, grounded on a strategic Provincial Water Resources master plan to be implemented at all levels.
In addition, one of the objectives of the BWRP project is to reinforce the capacities of the educational and research institutions in Balochistan to provide suitable agro-technology training and education, including conducting applied research on agro-ecological livestock and agriculture production.

\section*{Objectives}
The main objective of the short-term international consultancy assignment is to:
\begin{itemize}
    \item Review and evaluate the multi-criteria being employed to rank and select the basins, sub-basins, and watersheds/catchments to implement new/improved techniques and systems at large scale for increasing groundwater recharge, improved rangeland management, and low water intensity agriculture;
    \item Identify and suggest the most appropriate improvements in the multi-criteria or its weightage for objective ranking and selection of watersheds/catchments to implement these new/improved techniques and systems at large scale for increasing groundwater recharge, improved rangeland management and low water intensity agriculture; and
    \item Hold a stakeholder’s workshop for representatives from the community, local government (district/tehsil / union council) and the TA Team / FAO to present findings, and recommendations.
\end{itemize}

\section*{Key Tasks}
The following key tasks are anticipated to be undertaken by the multicriteria evaluation consultant:
\begin{itemize}
    \item To devise a proper methodology for MCDA
    \item Study BWRP and its requirements for the selection of basins, sub-basins, and watersheds/catchments
    \item Review the multi-criteria along with the methodology being employed to evaluate basins, sub-basins, and watersheds/catchments
    \item Suggest improvements in the multi-criteria being employed to evaluate basins, sub-basins, and watersheds/catchments, if any
    \item Review and evaluate weightage according to recognized parameters/factors
    \item If appropriate, undertake sensitivity and uncertainty and rank
    \item Identify and rank prequalification of high-ranking basins, sub-basins, and watersheds/catchment
    \item Identify and rank prequalification of high-candidate watersheds
\end{itemize}

\section*{Deliverables}
The international multi-criteria consultant will be required to work in close collaboration with FAO and Landell Mills team of experts to accomplish the following deliverables
\begin{itemize}
    \item Prepare a comprehensive report after review and evaluation of the multicriteria being employed to rank and select the basins, sub-basins/catchments.
    \item Suggest improvements in the multicriteria or its weightage for objective ranking and selection of watersheds/catchments and prepare a handbook on the application of multi-criteria for selection of watersheds/catchment.
\end{itemize}